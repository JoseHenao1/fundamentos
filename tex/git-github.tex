\documentclass[a4paper,11pt]{article}
\usepackage[colorlinks=true,linkcolor=blue,allcolors=red]{hyperref}

\renewcommand\thefootnote{\textcolor{red}{\arabic{footnote}}}

\title{Git \& GitHub}
\author{\\\\ José David Henao Gallego}

\begin{document}
\maketitle
\tableofcontents

\section{Git}
Git es un sistema de control de versiones distribuidas de código abierto y libre
de diseñado para manejar todo, desde proyectos pequeños hasta muy grandes con
velocidad y eficiencia. Fue creado por Linus Torvalds en 2005 para el desarrollo
de la Con otros desarrolladores de kernel contribuyendo a su  Desarrollo. Desde
2005, Junio Hamano ha sido el mantenedor central de.

Nos permite leer la historia del proyecto, sus cambios y los pensamientos de la
persona que trabajó en ellos , nota al pie de página, aunque a veces no
escribimos buenos mensajes de confirmación, a veces no usamos más de una línea
para comunicar ni nuestros pensamientos ni el propósito de nuestros cambios. Si
algo anda mal, podemos usarlo para regresar en el momento en que todo estaba
funcionando bien. Con Git podemos también etiquetar un estado específico de
nuestro proyecto con un número de versión si pensamos que algunos trabajo está
hecho, confiable y listo para usar.

\ section { Github }
GitHub, Inc. es una corporación multinacional estadounidense que ofrece alojamiento para 
desarrollo de software y control de versiones usando Git \ cite { GITHUB: 1 }. Tiene muchos
características como pull request (PR), seguimiento de errores, administrador de tareas y wikis para 
cada proyecto. Es uno de los proveedores de hosting más importantes de forma gratuita y
proyectos de código abierto, ya que alberga proyectos muy importantes.

Con GitHub podemos mejorar nuestros propios proyectos y ayudar en diferentes, dando 
recibir comentarios, informar problemas y compartir conocimientos e ideas. Nosotros podemos usar
alojado servicios de integración continua como Travis CI con él que nos permite
para probar y enviar nuestras aplicaciones con confianza \ cite { TRAVIS: 1 }.

Git $ + $ GitHub nos ofrece una manera fácil de trabajar con muchas personas en el mismo proyecto
\ footnote {Desde que aprendí a usar Git y GitHub (GitHub o GitLab o
Bitbucket, a quién le importa) No he podido encontrar una forma diferente de trabajar con 
más de una persona en el mismo proyecto.}, no importa si estamos en 
diferentes países, podemos compartir nuestro código e ideas con todos en el mundo.
\end{document}
