\documentclass[a4paper,11pt]{article}
\usepackage[colorlinks=true,linkcolor=blue,allcolors=red]{hyperref}

\renewcommand\thefootnote{\textcolor{red}{\arabic{footnote}}}

\title{\LARGE \bf
    Git \& GitHub
}

\author{\\\\ Cristian Camilo Serna Betancur  \\\\
Marfa Alejandra Kleber Sierra \\\\ José David Henao Gallego}

\begin{document}
\maketitle
\tableofcontents

\section{Git}
Git es un sistema de control de versiones distribuidas de código abierto y 
libre diseñado para manejar todo, desde proyectos pequeños hasta muy grandes 
con velocidad y eficiencia\cite{GIT:1}. Fue creado por Linus Torvalds en 2005 
para el desarrollo del \emph{Kernel de Linux}, con otros desarrolladores del
kernel contribuyendo a su  Desarrollo. Desde 2005, Junio Hamano ha sido el 
mantenedor central\cite{GIT:2}.

Nos permite leer la historia del proyecto, sus cambios y los pensamientos de la
persona que trabajó en ello\footnote{Aunque a veces no escribimos buenos 
mensajes de confirmación, a veces no usamos más de una línea para comunicar ni 
nuestros pensamientos ni el propósito de nuestros cambios.}. Si algo anda mal, 
podemos usarlo para regresar en el momento en que todo estaba funcionando bien. 
Con Git podemos etiquetar un estado específico de nuestro proyecto con un 
número de versión si pensamos que el actual estado del proyecto, es confiable y 
esta listo para usar.

\section{Github}
GitHub, Inc. es una corporación multinacional estadounidense que ofrece 
alojamiento para desarrollo de software y control de versiones usando Git
\cite{GITHUB:1}. Tiene muchos características como pull request (PR), 
seguimiento de errores, administrador de tareas y wikis para cada proyecto. Es 
uno de los proveedores de hosting más importantes de proyectos de código 
abierto, ya que alberga unos muy importantes\footnote{Un claro ejemplo es el
\emph{Kernel de Linux}}.

Con GitHub podemos mejorar nuestros propios proyectos y ayudar en diferentes, 
dando y recibiendo comentarios, informando sobre problemas y compartiendo 
conocimientos e ideas. 

Git $+$ GitHub nos ofrece una manera fácil de trabajar con muchas personas en 
el mismo proyecto \footnote{Desde que aprendí a usar Git y GitHub (GitHub o 
GitLab o Bitbucket, a quién le importa) No he podido encontrar una forma 
diferente de trabajar con más de una persona en el mismo proyecto.}, no importa 
si estamos en diferentes países, podemos compartir nuestro código e ideas con 
todos en el mundo. 

\bibliography{git-github}{}
\bibliographystyle{plain}

\end{document}
